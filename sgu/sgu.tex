\documentclass{article}
\usepackage{ctex,longtable,amssymb,amsmath}

\begin{document}
		\title{SGU做题表格}
		\author{翁健\\were}
		\date{Last updated on \today}
		\maketitle
		\begin{longtable}{|p{1cm}|p{5cm}|p{5cm}|}
			\hline
			题号 & 题意 & 做法 \\
			\hline
			100 &
			a + b &
			不解释.\\
			\hline
			101 &
			给定几张骨牌,正反面有数字,牌面上数字相同的可以相邻放在一起,牌可以翻面,然后问是否存在一种方案,使得所有牌都用上. &
			把牌面上的数字看作点,把每张牌看作连接两面的点的边,做一遍欧拉路径.
			trick在于如果一个点没有连边就当这个点不存在,但是如果其他情况造成的图不连通是无解的.\\
			\hline
			102 &
			问一个数以内有多少个数字和它互质. &
			范围很小,可以暴力,如果想写个欧拉函数也不反对.\\
			\hline
			103 &
			给定一个源点一个汇点,每个点上有一个初始颜色,每个颜色会周期性蓝紫交替,当颜色相同时候边可以通行.
			可以在当前节点等待至颜色相同再通行,求最短路.&
			问题主要在怎么算交替,最短路模型很裸.每次先判是否当前颜色相同,如果颜色相同无需等待.
			如果不同就分别算下一次变色的时间,然后较小的就是等待时间,如果相同就继续算等待时间.
			一共算两次,如果还是相同说明这条边没用.\\
			\hline
			104 &
			给若干束花和若干个花瓶,花插进花瓶里面有个美观值,花要按照顺序插,问最大美观值之和,输方案. &
			因为花要按照顺序插,所以就是DP了,输方案就在转移的时候记录一下.\\
			\hline
			105 &
			问把$1..N$所有数字按照一个字符串展开,变成一个好长的数字.前$N$个这样的数字能被$3$整除的个数. &
			写个暴力找规律,我就不剧透了.\\
			\hline
			106 &
			给一条直线$Ax + By + C = 0$,然后给一个矩形的左上角$(x1,y1)$右下角$(x2,y2)$,
			问这条直线在这个矩形里面有多少个整点. &
			先特判直线不存在、直线水平、直线垂直三种情况.然后对直线的$A,B$做一遍扩展欧几里德,算出一组解$(x,y)$
			和$d=gcd(A,B)$,那么直线上的解就是$x\frac{-C}{d}$和$y\frac{-C}{d}$.
			每组解就是$x+i\frac{B}{d}$和$y+i\frac{A}{d}$,$i$是整数.
			然后就是求在$[x1,x2]$和$[y1,y2]$值域的限制下$i$在数轴上的整点个数了.\\
			\hline
			107 &
			求$N$位数里面,平方末尾是$987654321$的数字个数. &
			暴力出$9$位有多少个,然后$10$位先$*9$,$11$位开始每次加一个$0$.\\
			\hline
			108 &
			每个数字可以向后生成一个新的数字,法则是自己加上自己各个位上的数字.
			然后有些数字是没法通过这个方法生成出来的,称作“封闭数”.
			问$N$以内的封闭数个数,还有$K$次询问,每次问$N$以内第$s$大的封闭数是几. &
			筛法是很容易想到的,但是题目卡$4MB$的内存,所以要压位了乱搞.\\
			\hline
			109 &
			这是一个异常恶趣味的题目,你和一个人玩游戏,在一个$N*N$的棋盘上面,他从最左上角开始走,
			每次你可以规定他走一定的步数.然后你可以把他永远走不到的一些点给删掉,最后把他逼死在一个点上.
			每次走的步数不能小于$N$,并且用过的数字不能再用. &
			先让他走$N$步,把曼哈顿距离$>N$的都删掉(如果$N=2$这步省略).
			然后每次走奇数步,你会发现会从国际象棋棋盘上的黑格走到白格,白格走到黑格,
			所以只要删掉最右下角的一条对角线就可以了~最后对手就被你逼回最左上角了.\\
			\hline
			110 &
			给你一束光,$N$个球体,然后照射球体反射,输出前十次反射,如果不足十次就输出前若干次,多于十次就输个' etc.'. &
			其实这题二维和三维一样做,一束光用直线的参数方程会方便好多,先算个解析式的交点,
			然后枚举每一个球找一个在光射出点最近的点作作为入射点.
			射出方向就是原来方向的在法向量分量上反向,所以用点积搞搞就可以了~\\
			\hline
			111 &
			高精度开根号. &
			第一遍写了个二分,发现细节有点多,好再有java.
			第二次用的牛顿迭代,发现精度上也有点问题要判断一下才能过.\\
			\hline
			112 &
			求$a^b - b^a$. &
			高精度,java秒. \\
			\hline
			113 &
			判断一个数能否被表示成两个质数的乘积. &
			分解质因数.\\
			\hline
			114 &
			在数轴上找一个点,使得其它所有点到这个点的点权*距离之和最小. &
			懒得推方程了,于是写了一个整体三分,貌似精度有点问题,最后把精度放低反而过了.\\
			\hline
			115 &
			问2001年的某月某日是礼拜几,如果这个日子不存在输出Impossible. &
			枚举日期模拟.\\
			\hline
			116 &
			定义一个数字是超级质数,那么这个数字是质数,并且它在质数里面的rank也是质数.
			给你一个数$N$,输出$N$最少能被几个超级质数表示出来,无解输出0. &
			求出质数表然后做个背包.\\
			\hline
			117 &
			给你$N$个数字,然后问它们的$M$次里面有多少个是能被$K$整除的. &
			快速幂.我赶脚数据范围问题,应该不会爆int才对,我用了int64才给过.\\
			\hline
			118 &
			先算出$\sum_{i=1}^{N}\prod_{j=1}^{i} a_j$,然后迭代算各个位上的数字之和,直到变成一位数. &
			高精度,java秒.\\
			\hline
			119 &
			问所有能够使得$ax+by|N$的$(x,y)$在那些$(a',b')$下面也成立. &
			开始觉得又要拓展欧几里德了,后来发现想多了.$ax+by=kN$要有$a'x+b'y=k'N$,
			则$a'=\frac{k'}{k}a$和$b'=\frac{k'}{k}b$.所以是求相异的
			$(i*a\quad mod\quad N,i*b\quad mod\quad N)$对个数.\\
			\hline
			120 &
			一个正$N$边形,每个点按照顺时针标号,然后给定点的编号和坐标,还原这个多边形. &
			先确定几何中心,然后用向量旋转跑一圈.\\
			\hline
			121 &
			给一张图,不重边不自环,将边黑白染色,要求每个点如果度大于1,那么有至少有连有一条黑边一条白边. &
			逐个连通块找奇数度数的点进行交替染色,如果没有就找偶数度数的.\\
			\hline
			122 &
			给一张图,满足Ore性质,构造一条汉密尔顿回路. &
			因为满足Ore性质,所以是可做的.
			做法是先随便找一条链(越长越好,保证复杂度),然后把它翻成一条链.
			翻成一条链的方法形象地说就是翻成一个8字然后拉直---找两个相邻的点,链头与靠近链尾的点相邻,链尾与靠近链头的点相邻.
			因为满足Ore性质,所以我们不用担心这对点找不到,然后每次都这么做,做完在链上开个小口重新开始上面的步骤.
			最后算法复杂度是$N^2$的.\\
			\hline
			123 &
			Fibonacci数列的前缀和,貌似不要高精度. &
			...\\
			\hline
			124 &
			问一点与一个多边形的关系,这个多边形的边和坐标轴平行.
			关系有“在里面”、“在边上”和在外面三种. &
			和坐标轴平行这个性质我开始以为要充分利用---离散化了再floodfill,后来证明想多了.
			然后其实是有几何的方法的,找一条射线和这个多边形求交点,奇数个就在内,偶数个就在外.
			有几个非规范的情况,比如和一条边相切、和一个顶点相交
			(这个比较麻烦,自己画画图就会发现把一个顶点当成一个交点或者两个线段的两个交点都不对).
			貌似和坐标轴平行就是为了避免大量的不规范情况.
			然后我开始一度以为是我的代码写错了,最后改了一改射线的方向就过了,我用的方向是水平.\\
			\hline
			125 &
			有A B两个矩阵,B矩阵表示A矩阵中每个值四联通的边上,有多少个值比它大.
			要求用B矩阵随便还原出一个A矩阵来. &
			范围很小,所以搜索,边搜边判就能过.\\
			\hline
			126 &
			$(a,b)$两个数,不断进行小的*2,大的减去小的增量这个过程,问能否使得其中一方为零. &
			因为是*2,所以我觉得答案不可能很大,一定在log级别内,所以暴力模拟就可以了,然后如果暴力迭代次数太多,就break输出无解.\\
			\hline
			127 &
			一本目录,已经有了两页,然后每一页最多只能有K行数据,数据按照升序排好,如果两个相邻的数据的首数字不同要新开一页.
			问最后一共要用多少页. &
			开始$ans=2$,然后遇到一组首数字相同的就个数除以K取上整,细节有点多..\\
			\hline
			128 &
			给定$N$个点,每个点都是一个多边形的拐角$90^{\circ}$,每条边都和坐标轴平行,每个点都要用上,
			然后把这些点还原成一个多边形,多边形不自交. &
			因为每个都是拐角,所以构造方案是唯一的.然后就排序两次,对着一行或者一列里面间隔连边.
			然后判断多边形的自交,把线段分成横向纵向两种,随便判判就好.
			如果构成的不是一个也要输出无解,wa了N久.\\
			\hline
			129 &
			判断一条线段在一个凸多边形里面的长度. &
			先用凸包还原出这个凸多边形.
			接下来判交情况比较多,我开始一直木有一个优美的做法.
			然后叉姐给了一个做法,先用叉积判断交点,然后再用叉积判断线段和凸多边形的边的旋转方向,从而判断在里面还是在外面,最后两段逼近.\\
			\hline
			130 &
			N个点一周摆在一个圆上,然后问不相交连N条线有多少种连法. &
			随便递推一下就好.\\
			\hline
			131 &
			给一个$NM$的grid,然后问用$2*2$的L型和$1*2$的矩形,有多少种方法盖满. &
			状压DP,$F_{i,S}$表示前$i-1$行盖满,第$i$行的状态为$S$的方案数.
			转移是枚举$i-1$行的状态,然后dfs每个格子是填放什么,生成$i$行的状态.\\
			\hline
			132 &
			给一个grid,开始有一些点已经被占领了,问最少放多少个$1*2$的矩形使得不能再放$1*2$的矩形. &
			这题卡得有点久,主要原因是dp不熟练.
			做法是记录两行的状态,$F_{i,S,T}$代表第$i-1$行状态为$S$,第$i$行为$T$时候最少放几个.
			此时$S$应该不能再塞.
			转移就是以一行为公有的,然后用原有的占领点作为初始状态,dfs转移.
			dfs转移的时候我一直在纠结两个本来是满的状态拼起来就不满怎么办,然后发现自己少判半(?)种情况.\\
			\hline
			133 &
			问有多少个区间被其他区间完全严格包含,所谓严格包含就是对于$[l,r]$与$[l',r']$,
			有$l<l' \land r'<r$.&
			这题被坑了啊,开始以为是有多少组这个关系,还写了个平衡树,最后发现sort一下扫一遍T\_T.\\
			\hline
			134 &
			给一棵树,删掉一个点之后使得剩下的连通块最大的最小的点称作重心,问有几个重心,分别是谁. &
			先bfs求出每个点的$size_i$,然后就随便乱搞了.\\
			\hline
			135 &
			在平面上画N条直线,最多把平面划成几块. &
			规律题,我就不剧透了.\\
			\hline
			136 &
			给定一个多边形的中点,然后随便还原出来一个符合的多边形. &
			高斯消元的特殊情况,问题在于奇偶要特判,奇数一定有唯一解,偶数可能有矛盾或者有无穷解,然后随便构造. 
			两条边在一条直线啦~自交啦~什么都可以啦~\\
			\hline
			137 &
			给一个N和一个S,构造一个长度为N,和为S的序列,使得这个序列头-1尾+1之后,可以通过圆环旋转,转回原序. &
			考虑$A=1...0,B=0...1$两个序列,那么我们可以在这个序列里面按照取模均匀填入$S mod N$个1,
			使得头为0尾为1,那么我们要做的就是求出这个“均匀”的长度,$k(S\qquad mod\qquad N) \equiv N (modN)$.
			因为题目保证了$(N,S) = 1$所以这个方程一定有解,然后枚举求解构造就可以了,构造的时候把多余的值用
			$\lfloor \frac{S}{N} \rfloor $平铺.\\
			\hline
			139 &
			十五数码判解. &
			这题是有结论的,先把原数组一维蛇形展开,并且忽略空位0,求取逆序对,逆序对个数为奇数有解反之无解.\\
			\hline
			140 &
			给一个N维向量,问是否存在一个N维向量使得它们的点积在modB的情况下为给定的P. &
			把B也看做向量的一维,变成一个N+1元的不定方程组,每次把前面的值打包,当成二元的做N遍拓展欧几里得. \\
			\hline
			141 &
			给定一组$x_1$,$x_2$,$K$,$P$,
			求一组$N_1,N_2,P_1,P_2$使得$N_1*x_1-P_1*x_1+N_2*x2-P_2*x_2 = P$,
			其中$N_1+N_2+P_1+P_2=K$.&
			这题开始做的时候想歪了(不是那个方面..觉得要解一个不定方程和一个不定方程组.
			后来发现想多了,先讲$N_1,P_1$和$N_2,P_2$打包成$A$和$B$,用拓展欧几里得找出一组解,
			然后用gcd的变化量找出这个解系中找出一个$|A|+|B|$最小的,如果最小还是比$K$大,那么无解.
			剩下的步数平分,左右走浪费掉.
			如果不能平分,就在最小的解两边抖动,看是否能使得剩下的步数平分,如果不能就是无解.\\
			\hline
			142 &
			给定一棵树,求这棵树权最大的一个连通子树. &
			是树形DP么?我赶脚就是个贪心水水..\\
			\hline
			144 &
			AB两个人,在两个整点之间到达,先到的会等后到的Z分钟,问两个人会面的概率. &
			画个图发现是求一个两条直线围成的多边形在一个矩形内的面积.\\
			\hline
			145 &
			给定源汇,求一条简单路径第K最短路. &
			本来想用spfa多记几个状态就好了,但是发现显然不是这样的,因为求出来的不是简单路径,可能重复走来走去.
			所以正确做法是二分一个长度,然后暴力出小于等于这个值的路径,随便K条.
			如果不足就放大二分,足够就缩小范围.
			最后确定长度之后随便暴力出一条等于这个值的路径.\\
			\hline
			146 &
			一个人在环形跑道上跑步,分为$N$个时段,每个时段作匀速运动,
			问在结束的时候与起点的距离. &
			读入输出都是指明小数点后四位,并且时间和速度也都是整数,所以转换成整数来做,但是在强制类型转换的时候貌似有精度误差,我也不会调,乱抖动一下吧.
			P.S.貌似可以用fmod实数取模函数来搞?!\\
			\hline
			148 &
			恐怖分子要炸水库,每一个水库有一个蓄水量一个初始水量,上游的水库炸坏了水会流到下游来,从而涨破下游的水库,破损的水的水库里面的水会全部向下流.
			每一个水库炸毁有一个代价,要求最后的代价最小,输出方案. &
			最朴素的暴力,枚举要强制炸毁的第一个水库,然后贪心向后扫.然后可以观察到强制毁坏水库$i$的话,要毁坏的水库集合为
			$\{ i \} \cup \{ j | s_j - s_{i-1} \le l_j \}$,其中$l_j$是水库$j$的最大需水量,$s_j - s_{i-1}$是$i..j$的水库里面的总水量.
			然后移项得到$\{ i \} \cup \{ j | s_j - l_j \le s_{i-1} \}$,从而我们可以从大到小枚举$i$,将$s_j - l_j \le s_{i-1}$维护在一个堆中.\\
			\hline
			149 &
			给定一棵无向有权树,然后求每一个点能到的最远的点的距离. &
			两遍dfs,第一遍求取每节点能向下走走到的最远点和次远点.第二遍,想办法维护每个点走上去能走到的最远点,即维护一个一个集合,从根走到该点的路径上,
			这些点到该点的距离$+$它们能够走下去最远的点,在里面选一个最大的.\\
			\hline
			151 & 
			给定两边和一条中线,求三点,使得这三点构成的三角形符合. &
			具体怎么推大家都会,主要是精度比较坑,看了别人的代码才回调eps的.\\
			\hline
			152 &
			给定一群人中每个人的得票数,然后要求算每个人的得票率精确到整数(每个人可能上整可能下整),要求最后的得票率之和为100,
			构造一种上下整的方式,使之符合要求. &
			具体做法是每次截尾取整,顺带统计下截尾取整的得票余数,然后把余数积累到一定程度的时候就把某个人取上整.\\
			\hline
			154 &
			找最小的$N$,使得$N!$末尾恰好有$Q$个0. &
			二分答案然后统计因子五的个数.\\
			\hline
			155 &
			构造笛卡尔树(就是给定每个节点的pri构造一个treap. &
			sort之后线段树搞搞T\_T.\\
			\hline
			160 &
			给定N个数字和一个M,要求求出一个N个数字的子集,使得这个子集的积mod M的值最大. &
			迭代宽搜.\\
			\hline
			163 &
			给N个数字$a_i$,然后给定一个$b$,然后求取所有正的$a_i^b$的和. &
			b很小,可以暴力乘方,然后和0取个max.\\
			\hline
			168 &
			给一个原矩阵,然后要求求一个目标矩阵. &
			目标矩阵其实是个和下标有关的曼哈顿距离的形状,然后随便搞搞.\\
			\hline
			169 &
			求$K$位数里面,能被各个位乘积整除的数字的个数.&
			结论题,好久前写的了不记得结论了,貌似是只有$11111X$这样的数字才有意义,然后逐$1-9$证明一下就好.\\
			\hline
			170 &
			给定两个字符串,问A最少经过多少次交换能变成B. &
			因为只有两种字母所以维护个队列随便搞搞就可以了.\\
			\hline
			172 &
			学校要安排考试,有的考试不能被安排在同一天,问能不能两天考完. &
			二分图判定(为啥不是2-SAT).\\
			\hline
			174 &
			有好多线段,都和坐标轴平行,然后每次加入一条,问什么时候这些线段组成的曲线闭合(线段不会自交).&
			并查集.\\
			\hline
			175 &
			给定一个字符串加密的方法,然后问长度为N加密前在位置Q的字幕最终在哪里. &
			仔细观察发现加密是一个两支递归的过程,然后模拟的时候只要走一支,所以$logN$的复杂度可以保证.\\
			\hline
			179 &
			问合法的括号序列,字典序的下一个是什么.&
			找最左边一个能把左括号变成右括号的地方.\\
			\hline
			180 &
			找逆序对. &
			树状数组.\\
			181 &
			给一个数列的生成方法,然后问第$k$个是几.&
			mod的数字很小,所以找一下循环节就好,然后注意$\rho$形状的循环节.\\
			\hline
			184 &
			求一个序列经过M此区间翻转的结果. &
			splay搞搞吧,回头用块链写一次.\\
			\hline
			186 &
			这题题意是读不懂的.&
			所以自求多福.\\
			\hline
			187 & 
			每次反转一个区间.&
			splay搞搞.\\
			\hline
			190 &
			一个$N\times N$的grid,要铺满$2\times 2$的骨牌,然后问怎么铺. &
			二分图最大匹配.\\
			\hline
			193 &
			N个人围成圈,然后从1开始传球给后K个人,然后问有多少个K是可以回到1手里的. &
			暴力打表找规律,然后写个高精度.\\
			\hline
			194 &
			无源汇可行流. &
			建图有点纠结,上网找的$\rightarrow\_\leftarrow$\\
			\hline
			196 &
			给一个$N\times M$的矩阵,是一张图,然后求这个矩阵和它的转置矩阵的乘积的和.&
			经过观察可以发现,就是每个节点的度数和.\\
			\hline
			199 &
			双关键字的最长不下降子序列.&
			先排序排掉第一个关键字,就变成单关键字的最长不下降子序列了,然后再用二分的方法做一遍$NlogN$的最长不下降子序列.\\
			\hline
			550 &
			给一棵树,然后每次删掉一条边,输出这条边的边权,然后两边的边,size较小的一遍乘上这条边权,较大的加上这条边权,
			如果等分,就编号较小的一遍乘上这条边全,较大的加上这条边权.&
			这题我是用ETT做的,因为扒出一个子树比较方便,然后关于两边等分其实不要维护什么多余的域,只管暴力扫一遍两边的点就好.
			这样最后就是在均摊上面多个$\log$.\\
			\hline
		\end{longtable}

\end{document}